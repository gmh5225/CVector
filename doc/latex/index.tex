\hypertarget{index_Intro}{}\section{Intro}\label{index_Intro}
This is a relatively simple A\-N\-S\-I compliant C vector library with specific structures and functions for int's, double's and string's and support for all other types using a generic structure where the type is passed in as void$\ast$ and stored in a byte array (to avoid dereferencing void$\ast$ warnings and frequent casting) . The generic vector is very flexible and allows you to provide free and init functions if you like that it will call at appropriate times similar to the way C++ containers will call destructors.

Other modifiable parameters are at the top of vector.\-c 
\begin{DoxyPre}
size\_t VEC\_I\_START\_SZ = 50;
size\_t VEC\_D\_START\_SZ = 50;
size\_t VEC\_START\_SZ = 20;
size\_t VEC\_S\_START\_SZ = 20;\end{DoxyPre}



\begin{DoxyPre}#define \hyperlink{vector__i_8c_a88db9d722845b6a23eb29d044a0a3c98}{VECI\_ALLOCATOR(x)} ((x) * 2)
#define \hyperlink{vector__d_8c_af2c425102d9020ae35b894de0c7eadea}{VECD\_ALLOCATOR(x)} ((x) * 2)
#define VECS\_ALLOCATOR(x) ((x) * 2)
#define \hyperlink{vector__void_8c_acc6ff7ec59b6544f657758f724fb7a8b}{VEC\_ALLOCATOR(x)} ((x) * 2)
\end{DoxyPre}
 The allocator macros are used in all functions that increase the size by 1. In others (constructors, insert\-\_\-array, reserve) V\-E\-C\-\_\-\-X\-\_\-\-S\-T\-A\-R\-T\-\_\-\-S\-Z is the amount extra allocated.

With version 2.\-0 I've added \hyperlink{vector__template_8c}{vector\-\_\-template.\-c} and \hyperlink{vector__template_8h}{vector\-\_\-template.\-h} which are used to generate code for any type (that doesn't require individual allocation/freeing like vector\-\_\-s). It behaves exactly like \hyperlink{structvector__i}{vector\-\_\-i} (or d). This is preferable to using the generic vector for simple types and basic structures etc. since it's faster and clearer.

To use generate your own c and h file for a type just run\-: 
\begin{DoxyPre}
python3 generate\_code.py yourtype
\end{DoxyPre}


which will generate vector\-\_\-yourtype.\-c and vector\-\_\-yourtype.\-h

\hyperlink{structvector__short}{vector\-\_\-short} is an example of the process and how to add it to the testing.\hypertarget{index_Building}{}\section{Building}\label{index_Building}
I use premake so the command on linux is premake4 gmake which will generate a build directory. cd into that and run make or make config=release. I have not tried it on windows though it should work (well I'm not sure about C\-Unit ...).

There is no output of any kind, no errors or warnings.

It has been relatively well tested using Cunit tests which all pass. I've also run it under valgrind and there are no memory leaks.

valgrind --leak-\/check=yes ./vector


\begin{DoxyPre}
==6100== 
==6100== HEAP SUMMARY:
==6100==     in use at exit: 0 bytes in 0 blocks
==6100==   total heap usage: 4,957 allocs, 4,957 frees, 927,837 bytes allocated
==6100== 
==6100== All heap blocks were freed -- no leaks are possible
==6100== 
==6100== ERROR SUMMARY: 0 errors from 0 contexts (suppressed: 2 from 2)
--6100-- 
--6100-- used\_suppression:      2 dl-hack3-cond-1
==6100== 
==6100== ERROR SUMMARY: 0 errors from 0 contexts (suppressed: 2 from 2)
\end{DoxyPre}


You can probably get Cunit from your package manager but if you want to get the most up to date version of C\-Unit go here\-:

\href{http://cunit.sourceforge.net/index.html}{\tt http\-://cunit.\-sourceforge.\-net/index.\-html} \href{http://sourceforge.net/projects/cunit/}{\tt http\-://sourceforge.\-net/projects/cunit/}

I used version 2.\-1-\/2.\hypertarget{index_Usage}{}\section{Usage}\label{index_Usage}
To actually use the library just copy vector.\-c and vector.\-h to your project. Also copy in generated types to your project as well. To get a good idea of how to use the library and see it in action and how it should behave, look at \hyperlink{vector__tests_8c}{vector\-\_\-tests.\-c}\hypertarget{index_LICENSE}{}\section{L\-I\-C\-E\-N\-S\-E}\label{index_LICENSE}
C\-Vector is licensed under the M\-I\-T License. Copyright (c) 2011-\/2013 Robert Winkler

Permission is hereby granted, free of charge, to any person obtaining a copy of this software and associated documentation files (the \char`\"{}\-Software\char`\"{}), to deal in the Software without restriction, including without limitation the rights to use, copy, modify, merge, publish, distribute, sublicense, and/or sell copies of the Software, and to permit persons to whom the Software is furnished to do so, subject to the following conditions\-:

The above copyright notice and this permission notice shall be included in all copies or substantial portions of the Software.

T\-H\-E S\-O\-F\-T\-W\-A\-R\-E I\-S P\-R\-O\-V\-I\-D\-E\-D \char`\"{}\-A\-S I\-S\char`\"{}, W\-I\-T\-H\-O\-U\-T W\-A\-R\-R\-A\-N\-T\-Y O\-F A\-N\-Y K\-I\-N\-D, E\-X\-P\-R\-E\-S\-S O\-R I\-M\-P\-L\-I\-E\-D, I\-N\-C\-L\-U\-D\-I\-N\-G B\-U\-T N\-O\-T L\-I\-M\-I\-T\-E\-D T\-O T\-H\-E W\-A\-R\-R\-A\-N\-T\-I\-E\-S O\-F M\-E\-R\-C\-H\-A\-N\-T\-A\-B\-I\-L\-I\-T\-Y, F\-I\-T\-N\-E\-S\-S F\-O\-R A P\-A\-R\-T\-I\-C\-U\-L\-A\-R P\-U\-R\-P\-O\-S\-E A\-N\-D N\-O\-N\-I\-N\-F\-R\-I\-N\-G\-E\-M\-E\-N\-T. I\-N N\-O E\-V\-E\-N\-T S\-H\-A\-L\-L T\-H\-E A\-U\-T\-H\-O\-R\-S O\-R C\-O\-P\-Y\-R\-I\-G\-H\-T H\-O\-L\-D\-E\-R\-S B\-E L\-I\-A\-B\-L\-E F\-O\-R A\-N\-Y C\-L\-A\-I\-M, D\-A\-M\-A\-G\-E\-S O\-R O\-T\-H\-E\-R L\-I\-A\-B\-I\-L\-I\-T\-Y, W\-H\-E\-T\-H\-E\-R I\-N A\-N A\-C\-T\-I\-O\-N O\-F C\-O\-N\-T\-R\-A\-C\-T, T\-O\-R\-T O\-R O\-T\-H\-E\-R\-W\-I\-S\-E, A\-R\-I\-S\-I\-N\-G F\-R\-O\-M, O\-U\-T O\-F O\-R I\-N C\-O\-N\-N\-E\-C\-T\-I\-O\-N W\-I\-T\-H T\-H\-E S\-O\-F\-T\-W\-A\-R\-E O\-R T\-H\-E U\-S\-E O\-R O\-T\-H\-E\-R D\-E\-A\-L\-I\-N\-G\-S I\-N T\-H\-E S\-O\-F\-T\-W\-A\-R\-E. 